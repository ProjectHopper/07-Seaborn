\documentclass[SKL-MASTER.tex]{subfiles}
\begin{document}
	\Large
\section*{Using KMeans to Cluster Data}
\begin{itemize}
\item Clustering is a very useful technique. Often, we need to divide and conquer when taking
actions. 
\item Consider a list of potential customers for a business. A business might need to
group customers into cohorts, and then departmentalize responsibilities for these cohorts.
\item Clustering can help facilitate the clustering process.
\item KMeans is probably one of the most well-known clustering algorithms and, in a larger sense,
one of the most well-known unsupervised learning techniques.
\end{itemize}

\subsubsection*{Getting ready}
First, let's walk through some simple clustering, then we'll talk about how KMeans works.
Also, since we'll be doing some plotting, import matplotlib as shown.

\begin{framed}
	\begin{verbatim}
>>> from sklearn.datasets import make_blobs
>>> blobs, classes = make_blobs(500, centers=3)
>>
>>> import matplotlib.pyplot as plt
\end{verbatim}
\end{framed}
\subsection*{Implementation} % How to do it…
\begin{itemize}
\item We are going to walk through a simple example that clusters blobs of fake data. 
\item Then we'll talk
a little bit about how KMeans works to find the optimal number of blobs.
\item Looking at our blobs, we can see that there are three distinct clusters:

\end{itemize}


\begin{framed}
	\begin{verbatim}
>>> f, ax = plt.subplots(figsize=(7.5, 7.5))
>>> ax.scatter(blobs[:, 0], blobs[:, 1], color=rgb[classes])
>>> rgb = np.array(['r', 'g', 'b'])
>>> ax.set_title("Blobs")
\end{verbatim}
\end{framed}
%========================================================%
%%- Chapter 3
%%- 87
The output is as follows:

\begin{figure}[h!]
\centering
\includegraphics[width=0.7\linewidth]{images/SKL30-Blobs}

\end{figure}

Now we can use KMeans to find the centers of these clusters. In the first example, we'll work on the basis that there are three centers:
%========================================================%
%%- Building Models with Distance Metrics
%%- 88

\begin{framed}
\begin{verbatim}
>>> from sklearn.cluster import KMeans
>>> kmean = KMeans(n_clusters=3)
>>> kmean.fit(blobs)
KMeans(copy_x=True, init='k-means++', max_iter=300, n_clusters=3,
n_init=10, n_jobs=1, precompute_distances=True,
random_state=None, tol=0.0001, verbose=0)
\end{verbatim}
\end{framed}

\begin{framed}
\begin{verbatim}
>>> kmean.cluster_centers_
array([[ 0.47819567, 1.80819197],
       [ 0.08627847, 8.24102715],
       [ 5.2026125 , 7.86881767]])
>>> f, ax = plt.subplots(figsize=(7.5, 7.5))
>>> ax.scatter(blobs[:, 0], blobs[:, 1], color=rgb[classes])
>>> ax.scatter(kmean.cluster_centers_[:, 0],
kmean.cluster_centers_[:, 1], marker='*', s=250,
color='black', label='Centers')
>>> ax.set_title("Blobs")
>>> ax.legend(loc='best')
\end{verbatim}
\end{framed}
\begin{figure}
\centering
\includegraphics[width=0.7\linewidth]{images/SKL30-Blobs-centers}
\end{figure}

The following screenshot shows the output:
%========================================================%
%Chapter 3
%89
Other attributes are useful too. For instance, the \texttt{labels\_} attribute will produce the expected
label for each point:
\begin{framed}
	\begin{verbatim}
>>> kmean.labels_[:5]
array([1, 1, 2, 2, 1], dtype=int32)
\end{verbatim}
\end{framed}
We can check whether \texttt{kmean.labels\_} is the same as classes, but because KMeans has no
knowledge of the classes going in, it cannot assign the sample index values to both classes:

\begin{framed}
\begin{verbatim}
>>> classes[:5]
array([0, 0, 2, 2, 0])
\end{verbatim}
\end{framed}

Feel free to swap 1 and 0 in classes to see if it matches up with \texttt{labels\_}.
The transform function is quite useful in the sense that it will output the distance between
each point and centroid:
\begin{framed}
\begin{verbatim}
>>> kmean.transform(blobs)[:5]
array([[ 6.47297373, 1.39043536, 6.4936008 ],
[ 6.78947843, 1.51914705, 3.67659072],
[ 7.24414567, 5.42840092, 0.76940367],
[ 8.56306214, 5.78156881, 0.89062961],
[ 7.32149254, 0.89737788, 5.12246797]])
\end{verbatim}
\end{framed}
\subsection*{Implementation}
KMeans is actually a very simple algorithm that works to minimize the within-cluster sum of
square distances from the mean. We'll be minimizing the sum of squares yet again!

It does this by first setting a pre-specified number of clusters, K, and then alternating
between the following:
\begin{itemize}
	\item Assigning each observation to the nearest cluster
	\item Updating each centroid by calculating the mean of each observation assigned
	to this cluster
\end{itemize}
This happens until some specified criterion is met.
%========================================================%
%Building Models with Distance Metrics
%90
\end{document}