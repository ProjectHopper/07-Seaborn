%=========================================================================================== %
\subsubsection{2.1.2. The curse of dimensionality}
\begin{itemize}
\item If the data is only described by one feature, with values ranging from 0 to 1, with n train observations, new data will no further away than 1/n and the nearest neighbor decision rule will be efficient as soon as 1/n is small compared to the scale of between-class feature variations.

\item If the number of features is p, the number of training samples to pave the [0, 1] space with a between-point distance of d, is 1/d**p. This number scales exponentialy p, the dimensionality of the problem.

\item In other words, the prediction problem becomes much harder for high-dimensional data. This is called the curse of dimensionality and is the core problem that machine learning addresses.

\end{itemize}

%=========================================================================================== %
\end{document}
