\documentclass[SKL-MASTER.tex]{subfiles}

\begin{document}
\Large
\section*{Using Pipelines for multiple preprocessing steps}
%Pipelines are something I don't think about using often, but are useful.
Pipelines can be used to tie together many steps into one object. This allows for easier tuning
and better access to the configuration of the entire model, not just one of the steps.
\subsection*{Getting ready}
\begin{itemize}
\item This is the first section where we'll combine multiple data processing steps into a single step.
\item In scikit-learn, this is known as a Pipeline. 
\item In this section, we'll first deal with missing data
via imputation. \item However, after that, we'll scale the data to get a mean of zero and a standard
deviation of one.
\end{itemize}
Let's create a dataset that is missing some values, and then we'll look at how to create
a Pipeline:
%------------------------%
\begin{framed}
\begin{verbatim}
>>> from sklearn import datasets
>>> import numpy as np
>>> mat = datasets.make_spd_matrix(10)
>>> masking_array = np.random.binomial(1, .1, mat.shape).astype(bool)
>>> mat[masking_array] = np.nan
>>> mat[:4, :4]
array([[ 0.56716186, -0.20344151, nan, -0.22579163],
       [ nan, 1.98881836, -2.25445983, 1.27024191],
       [ 0.29327486, -2.25445983, 3.15525425, -1.64685403],
       [-0.22579163, 1.27024191, -1.64685403, 1.32240835]])
\end{verbatim}
\end{framed}
%---------------------------- %
Great, now we can create a Pipeline.
%===========================================================================================%
%Premodel Workflow
%26
\subsection*{How to do it}
Without Pipelines, the process will look something like the following:
%------------------------%
\begin{framed}
	\begin{verbatim}
>>> from sklearn import preprocessing
>>> impute = preprocessing.Imputer()
>>> scaler = preprocessing.StandardScaler()
>>> mat_imputed = impute.fit_transform(mat)
>>> mat_imputed[:4, :4]
array([[ 0.56716186, -0.20344151, -0.80554023, -0.22579163],
[ 0.04235695, 1.98881836, -2.25445983, 1.27024191],
[ 0.29327486, -2.25445983, 3.15525425, -1.64685403],
[-0.22579163, 1.27024191, -1.64685403, 1.32240835]])
>>> mat_imp_and_scaled = scaler.fit_transform(mat_imputed)
array([[ 2.235e+00, -6.291e-01, 1.427e-16, -7.496e-01],
[ 0.000e+00, 1.158e+00, -9.309e-01, 9.072e-01],
[ 1.068e+00, -2.301e+00, 2.545e+00, -2.323e+00],
[ -1.142e+00, 5.721e-01, -5.405e-01, 9.650e-01]])
\end{verbatim}
\end{framed}
%---------------------------- %
Notice that the prior missing value is 0. This is expected because this value was imputed
using the mean strategy, and scaling subtracts the mean.
Now that we've looked at a non-Pipeline example, let's look at how we can incorporate
a Pipeline:
\begin{framed}
	\begin{verbatim}
>>> from sklearn import pipeline
>>> pipe = pipeline.Pipeline([('impute', impute), ('scaler', scaler)])
\end{verbatim}
\end{framed}
%---------------------------- %
Take a look at the Pipeline. As we can see, Pipeline defines the steps that designate the
progression of methods:
%------------------------%
\begin{framed}
	\begin{verbatim}
>>> pipe
Pipeline(steps=[('impute', Imputer(axis=0, copy=True, missing_
values='NaN', strategy='mean', verbose=0)), ('scalar',
StandardScaler(copy=True, with_mean=True, with_std=True))])
\end{verbatim}
\end{framed}
%---------------------------- %
This is the best part; simply call the \texttt{fit\_transform} method on the pipe object.
These separate steps are completed in a single step:

%------------------------%
\begin{framed}
\begin{verbatim}
>>> new_mat = pipe.fit_transform(mat)
>>> new_mat [:4, :4]
array([[ 2.235e+00, -6.291e-01, 1.427e-16, -7.496e-01],
[ 0.000e+00, 1.158e+00, -9.309e-01, 9.072e-01],
[ 1.068e+00, -2.301e+00, 2.545e+00, -2.323e+00],
[ -1.142e+00, 5.721e-01, -5.405e-01, 9.650e-01]])
\end{verbatim}
\end{framed}
%---------------------------- %
%===========================================================================================%
%-- Chapter 1
% %- 27
% %- We can also confirm that the two different methods give the same result:
%------------------------%
\begin{framed}
\begin{verbatim}
>>> np.array_equal(new_mat, mat_imp_and_scaled)
True
Beautiful!
\end{verbatim}
\end{framed}
%---------------------------- %
Later in the book, we'll see just how powerful this concept is. It doesn't stop at preprocessing
steps. It can easily extend to dimensionality reduction as well, fitting different learning methods.
Dimensionality reduction is handled on it's own in the recipe Reducing dimensionality with PCA.
\subsubsection{Implementation}
As mentioned earlier, almost every scikit-learn has a similar interface. The important ones
that allow Pipelines to function are:
\begin{itemize}
\item \texttt{fit}
\item \texttt{transform}
\item \texttt{fit\_transform} (a convenience method)
\end{itemize}

To be specific, if a Pipeline has N objects, the first N-1 objects must implement both fit and
transform, and the Nth object must implement at least fit. If this doesn't happen, an error
will be thrown.
Pipeline will work correctly if these conditions are met, but it is still possible that not every
method will work properly. For example, pipe has a method, \texttt{inverse\_transform}, which
does exactly what the name entails. However, because the impute step doesn't have an
\texttt{inverse\_transform} method, this method call will fail:
\begin{framed}
\begin{verbatim}
>>> pipe.inverse_transform(new_mat)
AttributeError: 'Imputer' object has no attribute 'inverse_transform'
However, this is possible with the scalar object:
>>> scaler.inverse_transform(new_mat) [:4, :4]
array([[ 0.567, -0.203, -0.806, -0.226],
[ 0.042, 1.989, -2.254, 1.27 ],
[ 0.293, -2.254, 3.155, -1.647],
[-0.226, 1.27 , -1.647, 1.322]])
\end{verbatim}
\end{framed}
Once a proper Pipeline is set up, it functions almost exactly how you'd expect. It's a series
of for loops that fit and transform at each intermediate step, feeding the output to the
subsequent transformation.
%===========================================================================================%
% % - Premodel Workflow
% % - 28
To conclude this recipe, I'll try to answer the "why?" question. There are two main reasons:
\begin{enumerate}
\item The first reason is convenience. The code becomes quite a bit cleaner; instead of
calling fit and transform over and over, it is offloaded to sklearn.
\item The second, and probably the more important, reason is cross validation. Models
can become very complex. If a single step in Pipeline has tuning parameters, they
might need to be tested; with a single step, the code overhead to test the parameters
is low. 
\end{enumerate}
However, five steps with all of their respective parameters can become difficult
to test. Pipelines ease a lot of the burden.
\end{document}