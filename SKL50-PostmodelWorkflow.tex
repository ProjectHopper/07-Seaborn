Postmodel Workflow
This chapter will cover the following recipes:
ff K-fold cross validation
ff Automatic cross validation
ff Cross validation with ShuffleSplit
ff Stratified k-fold
ff Poor man's grid search
ff Brute force grid search
ff Using dummy estimators to compare results
ff Regression model evaluation
ff Feature selection
ff Feature selection on L1 norms
ff Persisting models with joblib
%=========================================================================== %
Introduction
Even though by design the chapters are unordered, you could argue by virtue of the art of
data science, we've saved the best for last.
For the most part, each recipe within this chapter is applicable to the various models we've
worked with. In some ways, you can think about this chapter as tuning the parameters and
features. Ultimately, we need to choose some criteria to determine the "best" model. We'll use
various measures to define best. This is covered in the Regression model evaluation recipe.
Then in the Cross validation with ShuffleSplit recipe, we will randomize the evaluation across
subsets of the data to help avoid overfitting.
www.it-ebooks.info
Postmodel Workflow
162
K-fold cross validation
In this recipe, we'll create, quite possibly, the most important post-model validation
exercise—cross validation. We'll talk about k-fold cross validation in this recipe. There are
several varieties of cross validation, each with slightly different randomization schemes.
K-fold is perhaps one of the most well-known randomization schemes.
Getting ready
We'll create some data and then fit a classifier on the different folds. It's probably worth
mentioning that if you can keep a holdout set, then that would be best. For example, we
have a dataset where N = 1000. If we hold out 200 data points, then use cross validation
between the other 800 points to determine the best parameters.
How to do it...
First, we'll create some fake data, then we'll examine the parameters, and finally, we'll look
at the size of the resulting dataset:
>>> N = 1000>>>
holdout = 200
>>> from sklearn.datasets import make_regression
>>> X, y = make_regression(1000, shuffle=True)
Now that we have the data, let's hold out 200 points, and then go through the fold scheme
like we normally would:
>>> X_h, y_h = X[:holdout], y[:holdout]
>>> X_t, y_t = X[holdout:], y[holdout:]
>>> from sklearn.cross_validation import KFold
K-fold gives us the option of choosing how many folds we want, if we want the values to be
indices or Booleans, if want to shuffle the dataset, and finally, the random state (this is mainly
for reproducibility). Indices will actually be removed in later versions. It's assumed to be True.
Let's create the cross validation object:
>>> kfold = KFold(len(y_t), n_folds=4)
Now, we can iterate through the k-fold object:
>>> output_string = "Fold: {}, N_train: {}, N_test: {}"
>>> for i, (train, test) in enumerate(kfold):
www.it-ebooks.info
Chapter 5
163
print output_string.format(i, len(y_t[train]), len(y_t[test]))
Fold: 0, N_train: 600, N_test: 200
Fold: 1, N_train: 600, N_test: 200
Fold: 2, N_train: 600, N_test: 200
Fold: 3, N_train: 600, N_test: 200
Each iteration should return the same split size.
How it works...
It's probably clear, but k-fold works by iterating through the folds and holds out 1/n_folds *
N, where N for us was len(y_t).
From a Python perspective, the cross validation objects have an iterator that can be accessed
by using the in operator. Often times, it's useful to write a wrapper around a cross validation
object that will iterate a subset of the data. For example, we may have a dataset that has
repeated measures for data points or we may have a dataset with patients and each patient
having measures.
We're going to mix it up and use pandas for this part:
>>> import numpy as np
>>> import pandas as pd
>>> patients = np.repeat(np.arange(0, 100, dtype=np.int8), 8)
>>> measurements = pd.DataFrame({'patient_id': patients,
'ys': np.random.normal(0, 1, 800)})
Now that we have the data, we only want to hold out certain customers instead of data points:
>>> custids = np.unique(measurements.patient_id)
>>> customer_kfold = KFold(custids.size, n_folds=4)
>>> output_string = "Fold: {}, N_train: {}, N_test: {}"
>>> for i, (train, test) in enumerate(customer_kfold):
train_cust_ids = custids[train]
training = measurements[measurements.patient_id.isin(
train_cust_ids)]
testing = measurements[~measurements.patient_id.isin(
train_cust_ids)]
www.it-ebooks.info
Postmodel Workflow
164
print output_string.format(i, len(training), len(testing))
Fold: 0, N_train: 600, N_test: 200
Fold: 1, N_train: 600, N_test: 200
Fold: 2, N_train: 600, N_test: 200
Fold: 3, N_train: 600, N_test: 200
Automatic cross validation
We've looked at the using cross validation iterators that scikit-learn comes with, but we can
also use a helper function to perform cross validation for use automatically. This is similar
to how other objects in scikit-learn are wrapped by helper functions, pipeline for instance.
Getting ready
First, we'll need to create a sample classifier; this can really be anything, a decision tree,
a random forest, whatever. For us, it'll be a random forest. We'll then create a dataset
and use the cross validation functions.
How to do it...
First import the ensemble module and we'll get started:
>>> from sklearn import ensemble
>>> rf = ensemble.RandomForestRegressor(max_features='auto')
Okay, so now, let's create some regression data:
>>> from sklearn import datasets
>>> X, y = datasets.make_regression(10000, 10)
Now that we have the data, we can import the cross_validation module and get access
to the functions we'll use:
>>> from sklearn import cross_validation
>>> scores = cross_validation.cross_val_score(rf, X, y)
>>> print scores
[ 0.86823874 0.86763225 0.86986129]
%============================================================================================ %
%www.it-ebooks.info
%Chapter 5
%165
How it works...
For the most part, this will delegate to the cross validation objects. One nice thing is that, the
function will handle performing the cross validation in parallel.
We can activate verbose mode play by play:
>>> scores = cross_validation.cross_val_score(rf, X, y, verbose=3,
cv=4)
[CV] no parameters to be set
[CV] no parameters to be set, score=0.872866 - 0.7s
[CV] no parameters to be set
[CV] no parameters to be set, score=0.873679 - 0.6s
[CV] no parameters to be set
[CV] no parameters to be set, score=0.878018 - 0.7s
[CV] no parameters to be set
[CV] no parameters to be set, score=0.871598 - 0.6s
[Parallel(n_jobs=1)]: Done 1 jobs | elapsed: 0.7s
[Parallel(n_jobs=1)]: Done 4 out of 4 | elapsed: 2.6s finished
As we can see, during each iteration, we scored the function. We also get an idea of how long
the model runs.
It's also worth knowing that we can score our function predicated on which kind of model we're
trying to fit. In other recipes, we've discussed how to create your own scoring function.
Cross validation with ShuffleSplit
ShuffleSplit is one of the simplest cross validation techniques. This cross validation technique
will simply take a sample of the data for the number of iterations specified.
Getting ready
ShuffleSplit is another cross validation technique that is very simple. We'll specify the total
elements in the dataset, and it will take care of the rest. We'll walk through an example of
estimating the mean of a univariate dataset. This is somewhat similar to resampling, but it'll
illustrate one reason why we want to use cross validation while showing cross validation.
www.it-ebooks.info
Postmodel Workflow
166
How to do it...
First, we need to create the dataset. We'll use NumPy to create a dataset, where we know the
underlying mean. We'll sample half of the dataset to estimate the mean and see how close it
is to the underlying mean:
>>> import numpy as np
>>> true_loc = 1000
>>> true_scale = 10
>>> N = 1000
>>> dataset = np.random.normal(true_loc, true_scale, N)
>>> import matplotlib.pyplot as plt
>>> f, ax = plt.subplots(figsize=(7, 5))
>>> ax.hist(dataset, color='k', alpha=.65, histtype='stepfilled');
>>> ax.set_title("Histogram of dataset");
>>> f.savefig("978-1-78398-948-5_06_06.png")
NumPy will give the following output:


%============================================================================================ %
%www.it-ebooks.info
%Chapter 5
%167
Now, let's take the first half of the data and guess the mean:
>>> from sklearn import cross_validation
>>> holdout_set = dataset[:500]
>>> fitting_set = dataset[500:]
>>> estimate = fitting_set[:N/2].mean()
>>> import matplotlib.pyplot as plt
>>> f, ax = plt.subplots(figsize=(7, 5))
>>> ax.set_title("True Mean vs Regular Estimate")
>>> ax.vlines(true_loc, 0, 1, color='r', linestyles='-', lw=5,
alpha=.65, label='true mean')
>>> ax.vlines(estimate, 0, 1, color='g', linestyles='-', lw=5,
alpha=.65, label='regular estimate')
>>> ax.set_xlim(999, 1001)
>>> ax.legend()
>>> f.savefig("978-1-78398-948-5_06_07.png")
We'll get the following output:
www.it-ebooks.info
Postmodel Workflow
168
Now, we can use ShuffleSplit to fit the estimator on several smaller datasets:
>>> from sklearn.cross_validation import ShuffleSplit
>>> shuffle_split = ShuffleSplit(len(fitting_set))
>>> mean_p = []
>>> for train, _ in shuffle_split:
mean_p.append(fitting_set[train].mean())
shuf_estimate = np.mean(mean_p)
>>> import matplotlib.pyplot as plt
>>> f, ax = plt.subplots(figsize=(7, 5))
>>> ax.vlines(true_loc, 0, 1, color='r', linestyles='-', lw=5,
alpha=.65, label='true mean')
>>> ax.vlines(estimate, 0, 1, color='g', linestyles='-', lw=5,
alpha=.65, label='regular estimate')
>>> ax.vlines(shuf_estimate, 0, 1, color='b', linestyles='-', lw=5,
alpha=.65, label='shufflesplit estimate')
>>> ax.set_title("All Estimates")
>>> ax.set_xlim(999, 1001)
>>> ax.legend(loc=3)
www.it-ebooks.info
Chapter 5
169
The output will be as follows:
As we can see, we got an estimate that was similar to what we expected, but we were able to
take many samples to get that estimate.
Stratified k-fold
In this recipe, we'll quickly look at stratified k-fold valuation. We've walked through different
recipes where the class representation was unbalanced in some manner. Stratified k-fold is
nice because its scheme is specifically designed to maintain the class proportions.
Getting ready
We're going to create a small dataset. In this dataset, we will then use stratified k-fold validation.
We want it small so that we can see the variation. For larger samples. it probably won't be as big
of a deal.
We'll then plot the class proportions at each step to illustrate how the class proportions
are maintained:
>>> from sklearn import datasets
>>> X, y = datasets.make_classification(n_samples=int(1e3),
weights=[1./11])


%============================================================================================ %
%www.it-ebooks.info
%Postmodel Workflow
%170
Let's check the overall class weight distribution:
>>> y.mean()
0.90300000000000002
Roughly, 90.5 percent of the samples are 1, with the balance 0.
How to do it...
Let's create a stratified k-fold object and iterate it through each fold. We'll measure the
proportion of verse that are 1. After that we'll plot the proportion of classes by the split
number to see how and if it changes. This code will hopefully illustrate how this is beneficial.
We'll also plot this code against a basic ShuffleSplit:
>>> from sklearn import cross_validation
>>> n_folds = 50
>>> strat_kfold = cross_validation.StratifiedKFold(y,
n_folds=n_folds)
>>> shuff_split = cross_validation.ShuffleSplit(n=len(y),
n_iter=n_folds)
>>> kfold_y_props = []
>>> shuff_y_props = []
>>> for (k_train, k_test), (s_train, s_test) in zip(strat_kfold,
>>> shuff_split):
kfold_y_props.append(y[k_train].mean())
shuff_y_props.append(y[s_train].mean())
Now, let's plot the proportions over each fold:
>>> import matplotlib.pyplot as plt
>>> f, ax = plt.subplots(figsize=(7, 5))
>>> ax.plot(range(n_folds), shuff_y_props, label="ShuffleSplit",
color='k')
>>> ax.plot(range(n_folds), kfold_y_props, label="Stratified",
color='k', ls='--')
>>> ax.set_title("Comparing class proportions.")
>>> ax.legend(loc='best')

%============================================================================================ %
%www.it-ebooks.info
%Chapter 5
%171
The output will be as follows:
We can see that the proportion of each fold for stratified k-fold is stable across folds.
How it works...
Stratified k-fold works by taking the y value. First, getting the overall proportion of the classes,
then intelligently splitting the training and test set into the proportions. This will generalize to
multiple labels:
>>> import numpy as np
>>> three_classes = np.random.choice([1,2,3], p=[.1, .4, .5],
size=1000)
>>> import itertools as it
>>> for train, test in cross_validation.StratifiedKFold(three_classes, 5):
print np.bincount(three_classes[train])
[ 0 90 314 395]
[ 0 90 314 395]
www.it-ebooks.info
Postmodel Workflow
172
[ 0 90 314 395]
[ 0 91 315 395]
[ 0 91 315 396]
As we can see, we got roughly the sample sizes of each class for our training and
testing proportions.
Poor man's grid search
In this recipe, we're going to introduce grid search with basic Python, though we will use
sklearn for the models and matplotlib for the visualization.
Getting ready
In this recipe, we will perform the following tasks:
ff Design a basic search grid in the parameter space
ff Iterate through the grid and check the loss/score function at each point
in the parameter space for the dataset
ff Choose the point in the parameter space that minimizes/maximizes the
evaluation function
Also, the model we'll fit is a basic decision tree classifier. Our parameter space will be 2
dimensional to help us with the visualization:
The parameter space will then be the Cartesian product of the those two sets:
We'll see in a bit how we can iterate through this space with itertools.
Let's create the dataset and then get started:
>>> from sklearn import datasets
>>> X, y = datasets.make_classification(n_samples=2000, n_features=10)

%============================================================================================ %
%www.it-ebooks.info
%Chapter 5
%173
How to do it...
Earlier we said that we'd use grid search to tune two parameters—criteria and
max_features. We need to represent those as Python sets, and then use itertools
product to iterate through them:
>>> criteria = {'gini', 'entropy'}
>>> max_features = {'auto', 'log2', None}
>>> import itertools as it
>>> parameter_space = it.product(criteria, max_features)
Great! So now that we have the parameter space, let's iterate through it and check the accuracy
of each model as specified by the parameters. Then, we'll store that accuracy so that we can
compare different parameter spaces. We'll also use a test and train split of 50, 50:
import numpy as np
train_set = np.random.choice([True, False], size=len(y))
from sklearn.tree import DecisionTreeClassifier
accuracies = {}
for criterion, max_feature in parameter_space:
dt = DecisionTreeClassifier(criterion=criterion,
max_features=max_feature)
dt.fit(X[train_set], y[train_set])
accuracies[(criterion, max_feature)] = (dt.predict(X[~train_set])
== y[~train_set]).mean()
>>> accuracies
{('entropy', None): 0.974609375, ('entropy', 'auto'): 0.9736328125,
('entropy', 'log2'): 0.962890625, ('gini', None): 0.9677734375, ('gini',
'auto'): 0.9638671875, ('gini', 'log2'): 0.96875}
So we now have the accuracies and its performance. Let's visualize the performance:
>>> from matplotlib import pyplot as plt
>>> from matplotlib import cm
>>> cmap = cm.RdBu_r
>>> f, ax = plt.subplots(figsize=(7, 4))
>>> ax.set_xticklabels([''] + list(criteria))
>>> ax.set_yticklabels([''] + list(max_features))
>>> plot_array = []
>>> for max_feature in max_features:
www.it-ebooks.info
Postmodel Workflow
174
m = []
>>> for criterion in criteria:
m.append(accuracies[(criterion, max_feature)])
plot_array.append(m)
>>> colors = ax.matshow(plot_array, vmin=np.min(accuracies.values()) -
0.001, vmax=np.max(accuracies.values()) + 0.001, cmap=cmap)
>>> f.colorbar(colors)
The following is the output:
It's fairly easy to see which one performed best here. Hopefully, you can see how this process
can be taken to the further stage with a brute force method.
How it works...
This works fairly simply, we just have to perform the following steps:
1. Choose a set of parameters.
2. Iterate through them and find the accuracy of each step.
3. Find the best performer by visual inspection.

%============================================================================================ %
%www.it-ebooks.info
%Chapter 5
%175
\section{Brute force grid search}
In this recipe, we'll do an exhaustive grid search through scikit-learn. This is basically the same
thing we did in the previous recipe, but we'll utilize built-in methods.
We'll also walk through an example of performing randomized optimization. This is an alternative
to brute force search. Essentially, we're trading computer cycles to make sure that we search
the entire space. We were fairly calm in the last recipe. However, you could imagine a model
that has several steps, first imputation for fix missing data, then PCA reduce the dimensionality
to classification. Your parameter space could get very large, very fast; therefore, it can be
advantageous to only search a part of that space.
Getting ready
To get started, we'll need to perform the following steps:
1. Create some classification data.
2. We'll then create a LogisticRegression object that will be the model we're fitting.
3. After that, we'll create the search objects, GridSearch and RandomizedSearchCV.
%------------------------------------------------ %
How to do it...
Run the following code to create some classification data:
>>> from sklearn.datasets import make_classification
>>> X, y = make_classification(1000, n_features=5)
Now, we'll create our logistic regression object:
>>> from sklearn.linear_model import LogisticRegression
>>> lr = LogisticRegression(class_weight='auto')
We need to specify the parameters we want to search. For GridSearch, we can just specify
the ranges that we care about, but for RandomizedSearchCV, we'll need to actually specify
the distribution over the same space from which to sample:
>>> lr.fit(X, y)
LogisticRegression(C=1.0, class_weight={0: 0.25, 1: 0.75}, dual=False,
fit_intercept=True, intercept_scaling=1,
penalty='l2', random_state=None, tol=0.0001)
www.it-ebooks.info
Postmodel Workflow
176
>>> grid_search_params = {'penalty': ['l1', 'l2'],
'C': [1, 2, 3, 4]}
The only change we'll need to make is to describe the C parameter as a probability distribution.
We'll keep it simple right now, though we will use scipy to describe the distribution:
>>> import scipy.stats as st
>>> import numpy as np
>>> random_search_params = {'penalty': ['l1', 'l2'],
'C': st.randint(1, 4)}
How it works...
Now, we'll fit the classifier. This works by passing lr to the parameter search objects:
>>> from sklearn.grid_search import GridSearchCV, RandomizedSearchCV
>>> gs = GridSearchCV(lr, grid_search_params)
GridSearchCV implements the same API as the other models:
>>> gs.fit(X, y)
GridSearchCV(cv=None, estimator=LogisticRegression(C=1.0,
class_weight='auto', dual=False, fit_intercept=True,
intercept_scaling=1, penalty='l2', random_state=None,
tol=0.0001), fit_params={}, iid=True, loss_func=None,
n_jobs=1, param_grid={'penalty': ['l1', 'l2'],
'C': [1, 2, 3, 4]}, pre_dispatch='2*n_jobs', refit=True,
score_func=None, scoring=None, verbose=0)
As we can see with the param_grid parameter, our penalty and C are both arrays.
%========================================================================================= %
%www.it-ebooks.info
%Chapter 5
%177
To access the scores, we can use the grid_scores_ attribute of the grid search. We also
want to find the optimal set of parameters. We can also look at the marginal performance
of the grid search:
>>> gs.grid_scores_
[mean: 0.90300, std: 0.01192, params: {'penalty': 'l1', 'C': 1},
mean: 0.90100, std: 0.01258, params: {'penalty': 'l2', 'C': 1},
mean: 0.90200, std: 0.01117, params: {'penalty': 'l1', 'C': 2},
mean: 0.90100, std: 0.01258, params: {'penalty': 'l2', 'C': 2},
mean: 0.90200, std: 0.01117, params: {'penalty': 'l1', 'C': 3},
mean: 0.90100, std: 0.01258, params: {'penalty': 'l2', 'C': 3},
mean: 0.90100, std: 0.01258, params: {'penalty': 'l1', 'C': 4},
mean: 0.90100, std: 0.01258, params: {'penalty': 'l2', 'C': 4}]
We might want to get the max score:
>>> gs.grid_scores_[1][1]
0.90100000000000002
>>> max(gs.grid_scores_, key=lambda x: x[1])
mean: 0.90300, std: 0.01192, params: {'penalty': 'l1', 'C': 1}
The parameters obtained are the best choices for our logistic regression.
Using dummy estimators to compare results
This recipe is about creating fake estimators; this isn't the pretty or exciting stuff, but it is
worthwhile to have a reference point for the model you'll eventually build.
\subsection*{Getting ready}
In this recipe, we'll perform the following tasks:
1. Create some data random data.
2. Fit the various dummy estimators.
We'll perform these two steps for regression data and classification data.
%===================================================================== %
%www.it-ebooks.info
%Postmodel Workflow
%178
How to do it...
First, we'll create the random data:
>>> from sklearn.datasets import make_regression, make_classification
# classification if for later
>>> X, y = make_regression()
>>> from sklearn import dummy
>>> dumdum = dummy.DummyRegressor()
>>> dumdum.fit(X, y)
DummyRegressor(constant=None, strategy='mean')
By default, the estimator will predict by just taking the mean of the values and predicting the
mean values:
>>> dumdum.predict(X)[:5]
array([ 2.23297907, 2.23297907, 2.23297907, 2.23297907,
2.23297907])
There are other two other strategies we can try. We can predict a supplied constant (refer to
constant=None from the preceding command). We can also predict the median value.
Supplying a constant will only be considered if strategy is "constant".
Let's have a look:
>>> predictors = [("mean", None),
("median", None),
("constant", 10)]
>>> for strategy, constant in predictors:
dumdum = dummy.DummyRegressor(strategy=strategy,
constant=constant)
>>> dumdum.fit(X, y)
>>> print "strategy: {}".format(strategy), ",".join(map(str,
dumdum.predict(X)[:5]))
strategy: mean 2.23297906733,2.23297906733,2.23297906733,2.23297906733,2
.23297906733
%www.it-ebooks.info
%Chapter 5
%179
strategy: median 20.38535248,20.38535248,20.38535248,20.38535248,20.38535
248
strategy: constant 10.0,10.0,10.0,10.0,10.0
We actually have four options for classifiers. These strategies are similar to the continuous case,
it's just slanted toward classification problems:
>>> predictors = [("constant", 0),
("stratified", None),
("uniform", None),
("most_frequent", None)]
We'll also need to create some classification data:
>>> X, y = make_classification()
>>> for strategy, constant in predictors:
dumdum = dummy.DummyClassifier(strategy=strategy,
constant=constant)
dumdum.fit(X, y)
print "strategy: {}".format(strategy), ",".join(map(str,
dumdum.predict(X)[:5]))
strategy: constant 0,0,0,0,0
strategy: stratified 1,0,0,1,0
strategy: uniform 0,0,0,1,1
strategy: most_frequent 1,1,1,1,1
%============================================================================ %
How it works...
It's always good to test your models against the simplest models and that's exactly what
the dummy estimators give you. For example, imagine a fraud model. In this model, only
5 percent of the data set is fraud. Therefore, we can probably fit a pretty good model just
by never guessing any fraud.
We can create this model by using the stratified strategy, using the following command.
We can also get a good example of why class imbalance causes problems:
>>> X, y = make_classification(20000, weights=[.95, .05])
>>> dumdum = dummy.DummyClassifier(strategy='most_frequent')
>>> dumdum.fit(X, y)
%www.it-ebooks.info
%Postmodel Workflow
%180
DummyClassifier(constant=None, random_state=None, strategy='most_
frequent')
>>> from sklearn.metrics import accuracy_score
>>> print accuracy_score(y, dumdum.predict(X))
0.94575
We were actually correct very often, but that's not the point. The point is that this is our
baseline. If we cannot create a model for fraud that is more accurate than this, then it
isn't worth our time.
Regression model evaluation
We learned about quantifying the error in classification, now we'll discuss quantifying the error
for continuous problems. For example, we're trying to predict an age, not a gender.
Getting ready
Like the classification, we'll fake some data, then plot the change. We'll start simple, then
build up the complexity. The data will be a simulated linear model:
m = 2
b = 1
y = lambda x: m*x+b
Also, let's get our modules loaded:
>>> import numpy as np
>>> import matplotlib.pyplot as plt
>>> from sklearn import metrics
How to do it...
We will be performing the following actions:
1. Use 'y' to generate 'y_actual'.
2. Use 'y_actual' plus some err to generate 'y_prediction'.
3. Plot the differences.
4. Walk through various metrics and plot some of them.
%www.it-ebooks.info
%Chapter 5
%181
Let's take care of steps 1 and 2 at the same time and just have a function do the work for us.
This will be almost the same thing we just saw, but we'll add the ability to specify an error
(or bias if a constant):
>>> def data(x, m=2, b=1, e=None, s=10):
"""
Args:
x: The x value
m: Slope
b: Intercept
e: Error, optional, True will give random error
"""
if e is None:
e_i = 0
elif e is True:
e_i = np.random.normal(0, s, len(xs))
else:
e_i = e
return x * m + b + e_i
Now that we have the function, let's define y_hat and y_actual. We'll do it in a
convenient way:
>>> from functools import partial
>>> N = 100
>>> xs = np.sort(np.random.rand(N)*100)
>>> y_pred_gen = partial(data, x=xs, e=True)
>>> y_true_gen = partial(data, x=xs)
>>> y_pred = y_pred_gen()
>>> y_true = y_true_gen()
>>> f, ax = plt.subplots(figsize=(7, 5))
>>> ax.set_title("Plotting the fit vs the underlying process.")
>>> ax.scatter(xs, y_pred, label=r'$\hat{y}$')
%www.it-ebooks.info
%Postmodel Workflow
%182
>>> ax.plot(xs, y_true, label=r'$y$')
>>> ax.legend(loc='best')
The output for this code is as follows:
Just to confirm the output, we'd be working with the classical residuals:
>>> e_hat = y_pred - y_true
>>> f, ax = plt.subplots(figsize=(7, 5))
>>> ax.set_title("Residuals")
>>> ax.hist(e_hat, color='r', alpha=.5, histtype='stepfilled')
%www.it-ebooks.info
%Chapter 5
%183
The output for the residuals is as follows:
So that looks good now.
%============================================== %
How it works...
Now let's move to the metrics.
First, a metric is the mean squared error:
( ) (( )2 ) , trus pred trus pred MSE y y = E y − y
You can use the following code to find the value of the mean squared error:
>>> metrics.mean_squared_error(y_true, y_pred)
93.342352628475368
You'll notice that this code will penalize large errors more than small errors. It's important to
remember that all we're doing here is applying what probably was the cost function for the
model on the test data.
www.it-ebooks.info
Postmodel Workflow
184
Another option is the mean absolute deviation. We need to take the absolute value of the
difference, if we don't, our value will probably be fairly close to zero, the mean of the distribution:
( , ) (| |) trus pred trus pred MAD y y = E y − y
The final option is R2, this is 1 minus the ratio of squared errors for the overall mean and the
fit model. As the ratio tends to 0, the R2 tends to 1:
>>> metrics.r2_score(y_true, y_pred)
0.9729312117010761
R2 is deceptive; it cannot give the clearest sense of the accuracy of the model.
%=========================================================================== %
\section*{Feature selection}
This recipe along with the two following it will be centered around automatic feature selection.
I like to think of this as the feature analogue of parameter tuning. In the same way that we
cross-validate to find an appropriately general parameter, we can find an appropriately general
subset of features. This will involve several different methods.
The simplest idea is univariate selection. The other methods involve working with a combination
of features.
An added benefit to feature selection is that it can ease the burden on the data collection.
Imagine that you have built a model on a very small subset of the data. If all goes well, you
might want to scale up to predict the model on the entire subset of data. If this is the case,
you can ease the engineering effort of data collection at that scale.
Getting ready
With univariate feature selection, scoring functions will come to the forefront again. This time,
they will define the comparable measure by which we can eliminate features.
In this recipe, we'll fit a regression model with a few 10,000 features, but only 1,000 points.
We'll walk through the various univariate feature selection methods:
>>> from sklearn import datasets
>>> X, y = datasets.make_regression(1000, 10000)
%www.it-ebooks.info
%Chapter 5
%185
Now that we have the data, we will compare the features that are included with the various
methods. This is actually a very common situation when you're dealing in text analysis or
some areas of bioinformatics.
How to do it...
First, we need to import the feature_selection module:
>>> from sklearn import feature_selection
>>> f, p = feature_selection.f_regression(X, y)
Here, f is the f score associated with each linear model fit with just one of the features.
We can then compare these features and based on this comparison, we can cull features.
p is also the p value associated with that f value.
In statistics, the p value is the probability of a value more extreme than the current value of
the test statistic. Here, the f value is the test statistic:
>>> f[:5]
array([ 1.06271357e-03, 2.91136869e+00, 1.01886922e+00,
2.22483130e+00, 4.67624756e-01])
>>> p[:5]
array([ 0.97400066, 0.08826831, 0.31303204, 0.1361235, 0.49424067])
As we can see, many of the p values are quite large. We would rather want that the p values
be quite small. So, we can grab NumPy out of our tool box and choose all the p values less
than .05. These will be the features we'll use for the analysis:
>>> import numpy as np
>>> idx = np.arange(0, X.shape[1])
>>> features_to_keep = idx[p < .05]
>>> len(features_to_keep)
501
As you can see, we're actually keeping a relatively large amount of features. Depending on the
context of the model, we can tighten this p value. This will lessen the number of features kept.
Another option is using the VarianceThreshold object. We've learned a bit about it,
but it's important to understand that our ability to fit models is largely based on the variance
created by features. If there is no variance, then our features cannot describe the variation
in the dependent variable. A nice feature of this, as per the documentation, is that because
it does not use the outcome variable, it can be used for unsupervised cases.
%www.it-ebooks.info
%Postmodel Workflow
%186
We will need to set the threshold for which we eliminate features. In order to do that, we just
take the median of the feature variances and supply that:
>>> var_threshold = feature_selection.VarianceThreshold(np.median(np.
var(X, axis=1)))
>>> var_threshold.fit_transform(X).shape
(1000, 4835)
As we can see, we eliminated roughly half the features, more or less what we would expect.
How it works...
In general, all these methods work by fitting a basic model with a single feature. Depending on
whether we have a classification problem or a regression problem, we can use the appropriate
scoring function.
Let's look at a smaller problem and visualize how feature selection will eliminate certain
features. We'll use the same scoring function from the first example, but just 20 features:
>>> X, y = datasets.make_regression(10000, 20)
>>> f, p = feature_selection.f_regression(X, y)
Now let's plot the p values of the features, we can see which feature will be eliminated and
which will be kept:
>>> from matplotlib import pyplot as plt
>>> f, ax = plt.subplots(figsize=(7, 5))
>>> ax.bar(np.arange(20), p, color='k')
>>> ax.set_title("Feature p values")
www.it-ebooks.info
Chapter 5
187
The output will be as follows:
As we can see, many of the features won't be kept, but several will be.
Feature selection on L1 norms
We're going to work with some ideas similar to those we saw in the recipe on Lasso Regression.
In that recipe, we looked at the number of features that had zero coefficients.
Now we're going to take this a step further and use the spareness associated with L1 norms
to preprocess the features.
Getting ready
We'll use the diabetes dataset to fit a regression. First, we'll fit a basic LinearRegression
model with a ShuffleSplit cross validation. After we do that, we'll use LassoRegression
to find the coefficients that are 0 when using an L1 penalty. This hopefully will help us to avoid
overfitting, which means that the model is too specific to the data it was trained on. To put this
another way, the model, if overfit, does not generalize well to outside data.
We're going to perform the following steps:
1. Load the dataset.
2. Fit a basic linear regression model.
www.it-ebooks.info
Postmodel Workflow
188
3. Use feature selection to remove uninformative features.
4. Refit the linear regression and check to see how well it fits compared with the fully
featured model.
How to do it...
First, let's get the dataset:
>>> import sklearn.datasets as ds
>>> diabetes = ds.load_diabetes()
Let's create the LinearRegression object:
>>> from sklearn import linear_model
>>> lr = linear_model.LinearRegression()
Let's also import the metrics module for the mean_squared_error function and the
cross_validation module for the ShuffleSplit cross validation scheme:
>>> from sklearn import metrics
>>> from sklearn import cross_validation
>>> shuff = cross_validation.ShuffleSplit(diabetes.target.size)
Now, let's fit the model, and we'll keep track of the mean squared error for each iteration
of ShuffleSplit:
>>> mses = []
>>> for train, test in shuff:
train_X = diabetes.data[train]
train_y = diabetes.target[train]
test_X = diabetes.data[~train]
test_y = diabetes.target[~train]
lr.fit(train_X, train_y)
mses.append(metrics.mean_squared_error(test_y,
lr.predict(test_X)))
>>> np.mean(mses)
2856.366626198198
www.it-ebooks.info
Chapter 5
189
So now that we have the regular fit, let's check it after we eliminate any features with a zero
for the coefficient. Let's fit the Lasso Regression:
>>> from sklearn import feature_selection
>>> from sklearn import cross_validation
>>> cv = linear_model.LassoCV()
>>> cv.fit(diabetes.data, diabetes.target)
>>> cv.coef_
array([ -0. , -226.2375274 , 526.85738059, 314.44026013,
-196.92164002, 1.48742026, -151.78054083, 106.52846989,
530.58541123, 64.50588257])
We'll remove the first feature, I'll use a NumPy array to represent the columns that are to be
included in the model:
>>> import numpy as np
>>> columns = np.arange(diabetes.data.shape[1])[cv.coef_ != 0]
>>> columns
array([1, 2, 3 4, 5, 6, 7, 8, 9])
Okay, so now we'll fit the model with the specific features (see the columns in the following
code block):
>>> l1mses = []
>>> for train, test in shuff:
train_X = diabetes.data[train][:, columns]
train_y = diabetes.target[train]
test_X = diabetes.data[~train][:, columns]
test_y = diabetes.target[~train]
lr.fit(train_X, train_y)
l1mses.append(metrics.mean_squared_error(test_y,
lr.predict(test_X)))
>>> np.mean(l1mses)
2861.0763924492171
>>> np.mean(l1mses) - np.mean(mses)
4.7097662510191185
As we can see, even though we get an uninformative feature, the model still fits worse. This
isn't always the case. In the next section, we'll compare a fit between models where there are
many uninformative features.
www.it-ebooks.info
Postmodel Workflow
190
How it works...
First, we're going to create a regression dataset with many uninformative features:
>>> X, y = ds.make_regression(noise=5)
Let's fit a normal regression:
>>> mses = []
>>> shuff = cross_validation.ShuffleSplit(y.size)
>>> for train, test in shuff:
train_X = X[train]
train_y = y[train]
test_X = X[~train]
test_y = y[~train]
lr.fit(train_X, train_y)
mses.append(metrics.mean_squared_error(test_y,
lr.predict(test_X)))
>>> np.mean(mses)
879.75447864034209
Now, we can walk through the same process for Lasso regression:
>>> cv.fit(X, y)
LassoCV(alphas=None, copy_X=True, cv=None, eps=0.001,
fit_intercept=True, max_iter=1000, n_alphas=100,
n_jobs=1, normalize=False, positive=False, precompute='auto',
tol=0.0001, verbose=False)
We'll create the columns again. This is a nice pattern that will allow us to specify the features
we want to include:
>>> import numpy as np
>>> columns = np.arange(X.shape[1])[cv.coef_ != 0]
>>> columns[:5]
array([11, 15, 17, 20, 21,])
>>> mses = []
www.it-ebooks.info
Chapter 5
191
>>> shuff = cross_validation.ShuffleSplit(y.size)
>>> for train, test in shuff:
train_X = X[train][:, columns]
train_y = y[train]
test_X = X[~train][:, columns]
test_y = y[~train]
lr.fit(train_X, train_y)
mses.append(metrics.mean_squared_error(test_y,
lr.predict(test_X)))
>>> np.mean(mses)
15.755403220117708
As we can see, we get an extreme improvement in the fit of the model. This just exemplifies
that we need to be cognizant that not all the models need to be or should be thrown into
the model.
Persisting models with joblib
In this recipe, we're going to show how you can keep your model around for a later usage.
For example, you might want to actually use a model to predict the outcome and automatically
make a decision.
Getting ready
In this recipe, we will perform the following tasks:
1. Fit the model that we will persist.
2. Import joblib and save the model.
How to do it...
To persist models with joblib, the following code can be used:
>>> from sklearn import datasets, tree
>>> X, y = datasets.make_classification()
www.it-ebooks.info
Postmodel Workflow
192
>>> dt = tree.DecisionTreeClassifier()
>>> dt.fit(X, y)
DecisionTreeClassifier(compute_importances=None, criterion='gini',
max_depth=None, max_features=None,
max_leaf_nodes=None, min_density=None,
min_samples_leaf=1, min_samples_split=2,
random_state=None, splitter='best')
>>> from sklearn.externals import joblib
>>> joblib.dump(dt, "dtree.clf")
['dtree.clf',
'dtree.clf_01.npy',
'dtree.clf_02.npy',
'dtree.clf_03.npy',
'dtree.clf_04.npy']
How it works...
The preceding code works by saving the state of the object that can be reloaded into a
scikit-learn object. It's important to note that the state of model will have varying levels
of complexity, given the model type.
For simplicity sake, consider that all we'd need to save is the way to predict the outcome
for the given inputs. Well, for regression that would be easy, a little matrix algebra and
we're done. However, for models like random forest, where we could have many trees,
and those trees could be of various complexity levels, regression is difficult.
There's more...
We can check the size of decision tree versus random forest:
>>> from sklearn import ensemble
>>> rf = ensemble.RandomForestClassifier()
>>> rf.fit(X, y)
RandomForestClassifier(bootstrap=True, compute_importances=None,
criterion='gini', max_depth=None,
max_features='auto', max_leaf_nodes=None,
min_density=None, min_samples_leaf=1,
min_samples_split=2, n_estimators=10,
n_jobs=1, oob_score=False, random_state=None,
verbose=0)
www.it-ebooks.info
Chapter 5
193
I'm going to omit the output, but in total, there we were 52 files outputted on my machine:
>>> joblib.dump(rf, "rf.clf")
['rf.clf',
'rf.clf_01.npy',
'rf.clf_02.npy',
'rf.clf_03.npy',
'rf.clf_04.npy',
'rf.clf_05.npy',
'rf.clf_06.npy',…]
\end{document}