\documentclass[SKL-MASTER.tex]{subfiles}
\begin{document}
\LARGE

\noindent \textbf{Estimators objects: Fitting data:}\\The core object of scikit-learn is the estimator object. All estimator objects expose a \texttt{fit} method, that takes as input a dataset (2D array):

\begin{framed}
\begin{verbatim}
>>> estimator.fit(data)
\end{verbatim}
\end{framed}

\noindent Suppose \texttt{LogReg} and \texttt{KNN} are (shorthand names for) scikit-learn estimators.
\begin{framed}
\begin{verbatim}
>>> # Supervised Learning Problem
>>> LogReg.fit(SAheartFeat, SAheartTarget)
>>>
>>> # Unsupervised Learning Problem
>>> KNN.fit(IrisFeat)
\end{verbatim}
\end{framed}

\newpage
%------------------------------------------------%
\noindent \textbf{Estimator parameters:}\\
All the parameters of an estimator can be set when it is instanciated, or by modifying the corresponding attribute:

\begin{framed}
\begin{verbatim}
>>> estimator = Estimator(param1=1, param2=2)
>>> estimator.param1
\end{verbatim}
\end{framed}

%------------------------------------------------%
% % \subsubsection{Retrieving Estimator parameters:}

\noindent \textbf{Retrieving Estimator parameters:}\\ 
\begin{itemize}
\item When data is fitted with an estimator, parameters are estimated from the data at hand.
\item All the estimated parameters are attributes of the estimator object ending by an underscore:
\end{itemize}
\begin{framed}
\begin{verbatim}
>>> estimator.estimated_param_ 
\end{verbatim}
\end{framed}
%========================================================================%
\end{document}