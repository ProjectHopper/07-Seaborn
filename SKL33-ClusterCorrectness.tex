\documentclass[SKL-MASTER.tex]{subfiles}
\begin{document}
\Large
\section*{Assessing cluster correctness}
We talked a little bit about assessing clusters when the ground truth is not known. However,
we have not yet talked about assessing KMeans when the cluster is known. In a lot of cases,
this isn't knowable; however, if there is outside annotation, we will know the ground truth,
or at least the proxy, sometimes.
\subsection*{Getting Some Data}
So, let's assume a world where we have some outside agent supplying us with the
ground truth.
We'll create a simple dataset, evaluate the measures of correctness against the
ground truth in several ways, and then discuss them:
%--------------------- %
\begin{framed}
\begin{verbatim}
>>> from sklearn import datasets
>>> from sklearn import cluster
>>> blobs, ground_truth = datasets.make_blobs(1000, 
   centers=3, cluster_std=1.75)
\end{verbatim}
\end{framed}

% \subsubsection{Implementation}
\noindent Before we walk through the metrics, let's take a look at the dataset:
%--------------------- %
\begin{framed}
\begin{verbatim}
>>> f, ax = plt.subplots(figsize=(7, 5))
>>> colors = ['r', 'g', 'b']
>>> for i in range(3):
    p = blobs[ground_truth == i]
    ax.scatter(p[:,0], p[:,1],   c=colors[i],
    label="Cluster {}".format(i))
>>> ax.set_title("Cluster With Ground Truth")
>>> ax.legend()
>>> f.savefig("9485OS_03-16")
\end{verbatim}
\end{framed}
%========================================================%
% % Building Models with Distance Metrics
% % 94
The following is the output:
\begin{figure}[h!]
\centering
\includegraphics[width=0.7\linewidth]{Cluster1}
% \caption{}
% \label{fig:Cluster1}
\end{figure}

In order to fit a KMeans model we'll create a KMeans object from the cluster module:
%--------------------- %
\begin{framed}
\begin{verbatim}
>>> kmeans = cluster.KMeans(n_clusters=3)
>>> kmeans.fit(blobs)
KMeans(copy_x=True, init='k-means++', max_iter=300, 
       n_clusters=3, n_init=10, n_jobs=1,
       precompute_distances=True,
       random_state=None, tol=0.0001, verbose=0)
>>>
>>> kmeans.cluster_centers_
array([[  5.18993766,  0.35110059],
       [  0.18300097, -4.9480336 ],
       [ 10.01421381, -2.26274328]])
       \end{verbatim}
    \end{framed}
Now that we've fit the model, let's have a look at the cluster centroids:
%========================================================%
% % Chapter 3
% % 95
\begin{framed}
\begin{verbatim}
>>> f, ax = plt.subplots(figsize=(7, 5))
>>> colors = ['r', 'g', 'b']
>>> for i in range(3):
    p = blobs[ground_truth == i]
    ax.scatter(p[:,0], p[:,1], c=colors[i],
    label="Cluster {}".format(i))
>>> ax.scatter(kmeans.cluster_centers_[:, 0],
      kmeans.cluster_centers_[:, 1], 
      s=100, 
      color='black',
      label='Centers')
>>> ax.set_title("Cluster With Ground Truth")
>>> ax.legend()
>>> f.savefig("9485OS_03-17")
\end{verbatim}
\end{framed}
\noindent The following is the output: \textbf{GRAPH}
Now that we can view the clustering performance as a classification exercise, the metrics that
are useful in its context are also useful here:
%--------------------- %
{
\large
\begin{framed}
\begin{verbatim}
>>> for i in range(3):
		print (kmeans.labels_ == ground_truth)[ground_truth == i]
			.astype(int).mean()
0.0778443113772
0.990990990991
0.0570570570571
\end{verbatim}
\end{framed}
}
%========================================================%
% % Building Models with Distance Metrics
% % 96
Clearly, we have some backward clusters. So, let's get this straightened out first, and then
we'll look at the accuracy:
%--------------------- %
{
\large
\begin{framed}
\begin{verbatim}
>>> new_ground_truth = ground_truth.copy()
>>> new_ground_truth[ground_truth == 0] = 2
>>> new_ground_truth[ground_truth == 2] = 0
>>> for i in range(3):
		print (kmeans.labels_ == new_ground_truth)[ground_truth == i]
			.astype(int).mean()
0.919161676647
0.990990990991
0.90990990991
\end{verbatim}
\end{framed}
}
So, we're roughly correct 90 percent of the time. The second measure of similarity we'll look
at is the mutual information score:
%--------------------- %
\begin{framed}
\begin{verbatim}
>>> from sklearn import metrics
>>> metrics.normalized_mutual_info_score(ground_truth, kmeans.labels_)
0.78533737204433651
\end{verbatim}
\end{framed}
As the score tends to be 0, the label assignments are probably not generated through
similar processes; however, the score being closer to 1 means that there is a large
amount of agreement between the two labels.
For example, let's look at what happens when the mutual information score itself:
%--------------------- %
\begin{framed}
\begin{verbatim}
>>> metrics.normalized_mutual_info_score(ground_truth, 
ground_truth)
1.0
\end{verbatim}
\end{framed}
Given the name, we can tell that there is probably an unnormalized \texttt{mutual\_info\_score}:
%--------------------- %
\begin{framed}
\begin{verbatim}
>>> metrics.mutual_info_score(ground_truth, kmeans.labels_)
0.78945287371677486
\end{verbatim}
\end{framed}
These are very close; however, normalized mutual information is the mutual information
divided by the root of the product of the entropy of each set truth and assigned label.
%========================================================%
% % Chapter 3
% % 97
\subsection*{Further Remarks} % There's more...
\begin{itemize}
\item One cluster metric we haven't talked about yet and one that is not reliant on the ground truth
is inertia. 
\item It is not very well documented as a metric at the moment. 
\item However, it is the metric
that \texttt{KMeans} minimizes.
\end{itemize}
\subsection*{Inertia}
Inertia is the sum of the squared difference between each point and its assigned cluster.
We can use a little NumPy to determine this:
%--------------------- %
\begin{framed}
	\begin{verbatim}
>>> kmeans.inertia_
\end{verbatim}
\end{framed}

for scikit learn, inertia is calculated as the sum of squared distance for each point to it's closest centroid, i.e., its assigned cluster. So $I = \sum_{i}(d(i,cr))$ where cr is the centroid of the assigned cluster and d is the squared distance.
%
%Now the formula of gap statistic involves
%\[W_k = \sum_{r=1}^{k}\frac 1 {(2*n_r) }D_r\]
%where Dr is the sum of the squared distances between all points in cluster r.
%
%By introducing +c, −c in the squared distance formula (c being the centroid of cluster r coordinates), I have a term that corresponds to Inertia (as in scikit) + a term that disappears if each c is the barycentre of each cluster (which it is supposed to be in kmeans). So I guess Wk is in fact scikit Inertia.
\end{document}

